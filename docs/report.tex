\documentclass[a4paper,12pt]{report}

\usepackage{alltt, fancyvrb, url}
\usepackage{graphicx}
\usepackage[utf8]{inputenc}
\usepackage{float}
\usepackage{hyperref}
\usepackage{tikz}

% Questo commentalo se vuoi scrivere in inglese.
\usepackage[italian]{babel}

\usepackage[italian]{cleveref}

\title{Progetto di Programmazione ad Oggetti\\``The Exiled''}

\author{Luca Casadei, Francesco Pazzaglia, Marco Magnani, Manuel Baldoni}
\date{\today}


\begin{document}

\maketitle

\tableofcontents

\chapter{Analisi}

\section{Requisiti}
L'applicazione è un gioco che presenta un personaggio controllato dal giocatore con la possibilità di muoversi nella mappa in 4 direzioni e di combattere contro dei nemici utilizzando magie elementali. I nemici sconfitti potrebbero rilasciare delle cure o dei potenziamenti che favoriscono il giocatore. Per concludere il gioco, il giocatore deve entrare in possesso di 4 cristalli che vengono consegnati una volta sconfitti i 4 boss del gioco che sono nemici più difficili da sconfiggere.
\subsection*{Requisiti funzionali}
\begin{itemize}
    \item Movimento del giocatore.
    \item Movimento dei nemici.
    \item Presenza di oggetti ottenibili dal giocatore (Cure, potenziamenti e cristalli).
    \item Terminazione del gioco una volta raccolti 4 cristalli o se il giocatore viene sconfitto.
    \item Posizionamento e distribuzione degli oggetti.
    \item Possibilità di fare battaglie tra giocatore e nemici.
    \item Nemici più forti (boss) che se sconfitti rilasciano i cristalli per terminare il gioco.
    \item Possibilità di utilizzo di magie in battaglia di diverso tipo (Fuoco, Acqua, Fulmine, Erba).
    \item Aumento di livello tramite guadagno di esperienza sconfiggendo i nemici.
\end{itemize}

\subsection*{Requisiti non funzionali}
\begin{itemize}
    \item Il gioco deve essere multipiattaforma.
    \item \textit{TODO}
\end{itemize}

\section{Analisi e modello del dominio}

All'inizio al giocatore viene chiesto di che tipo elementale sarà il personaggio da controllare, in base a questa scelta gli verrà quindi assegnata una mossa di base di quel tipo e avrà le mosse di quel tipo potenziate fino alla fine del gioco.\\
Il giocatore può muoversi di una cella alla volta in una mappa a griglia in 4 direzioni (Nord, Est, Sud, Ovest) e una battaglia con i nemici (anch'essi di un certo tipo elementale) inizia quando la cella in cui è il giocatore combacia con quella di un nemico.\\
Per affrontare un nemico in battaglia il giocatore ha a disposizione massimo 4 mosse (non per forza dello stesso tipo del giocatore). Il combattimento avviene a turni alternati, il primo è il giocatore, poi il nemico e così via finché uno dei due viene sconfitto, se è il giocatore, il gioco termina.\\
Allo sconfiggere dei nemici viene conferita al giocatore una certa quantità di esperienza che serve ad aumentare di livello e un oggetto casuale, l'oggetto viene salvato nell'inventario del giocatore e potrà essere di diverso tipo, es. oggetto curativo(ripristina un tot di vita), booster di statistiche(per esempio aumenta l'attacco di un certo tipo di mosse) ecc.. L'aumento di livello comporta un incremento generale delle statistiche ovvero attacco, difesa e vita di un valore costante. Ogni 5 livelli verrà presentata al giocatore la possibilità di imparare una nuova mossa di un tipo casuale, se il giocatore ha già 4 mosse potrà decidere di scambiare quella nuova con una di quelle che conosce già. All'aumentare di livello sarà richiesta sempre più esperienza per passare a quello successivo.\\
Ci sono diverse classi di nemico, alcuni più deboli e altri più forti, come i boss, che se sconfitti consegnano uno dei 4 oggetti necessari per concludere il gioco (i cristalli).\\

\subsection*{Tipi elementali}
I seguenti sono i tipi elementali presenti, ognuno è efficace rispetto ad un altro, quindi quando vengono usate mosse di un certo tipo che risulta essere efficace rispetto al tipo di un nemico, si avranno dei moltiplicatori del danno. Lo stesso vale per i nemici che usano mosse di un tipo efficace rispetto a quello scelto dal giocatore all'inizio.
\\\\
Ad esempio, nel seguente schema si vede che Fulmine è efficace contro Acqua.


\begin{center}
\begin{tabular}{ c c c }
    Fulmine & $\rightarrow$ & Acqua \\
    Erba & $\rightarrow$ & Fulmine \\
    Acqua & $\rightarrow$ & Fuoco \\
    Fuoco & $\rightarrow$ & Erba \\
\end{tabular}
\end{center}

\subsection{UML}
\begin{figure}[H]
	\centering
	% generated by Plantuml 1.2023.13      
\definecolor{plantucolor0000}{RGB}{241,241,241}
\definecolor{plantucolor0001}{RGB}{24,24,24}
\definecolor{plantucolor0002}{RGB}{180,167,229}
\definecolor{plantucolor0003}{RGB}{0,0,0}
\definecolor{plantucolor0004}{RGB}{132,190,132}
\definecolor{plantucolor0005}{RGB}{3,128,72}
\definecolor{plantucolor0006}{RGB}{235,147,127}
\scalebox{0.5}{
\begin{tikzpicture}[yscale=-1
,pstyle0/.style={color=plantucolor0001,fill=plantucolor0000,line width=0.5pt}
,pstyle1/.style={color=plantucolor0001,fill=plantucolor0002,line width=1.0pt}
,pstyle2/.style={color=plantucolor0001,line width=0.5pt}
,pstyle3/.style={color=plantucolor0005,fill=plantucolor0004,line width=1.0pt}
,pstyle5/.style={color=plantucolor0001,line width=1.0pt}
,pstyle6/.style={color=plantucolor0001,fill=plantucolor0001,line width=1.0pt}
]
\draw[pstyle0] (232.02pt,155pt) arc (180:270:5pt) -- (237.02pt,150pt) -- (302.8795pt,150pt) arc (270:360:5pt) -- (307.8795pt,155pt) -- (307.8795pt,193pt) arc (0:90:5pt) -- (302.8795pt,198pt) -- (237.02pt,198pt) arc (90:180:5pt) -- (232.02pt,193pt) -- cycle;
\draw[pstyle1] (247.02pt,166pt) ellipse (11pt and 11pt);
\node at (247.02pt,166pt)[]{\textbf{\Large I}};
\node at (261.02pt,157.127pt)[below right,color=black]{\textit{Player}};
\draw[pstyle2] (233.02pt,182pt) -- (306.8795pt,182pt);
\draw[pstyle2] (233.02pt,190pt) -- (306.8795pt,190pt);
\draw[pstyle0] (96.52pt,155pt) arc (180:270:5pt) -- (101.52pt,150pt) -- (168.0674pt,150pt) arc (270:360:5pt) -- (173.0674pt,155pt) -- (173.0674pt,193pt) arc (0:90:5pt) -- (168.0674pt,198pt) -- (101.52pt,198pt) arc (90:180:5pt) -- (96.52pt,193pt) -- cycle;
\draw[pstyle1] (111.52pt,166pt) ellipse (11pt and 11pt);
\node at (111.52pt,166pt)[]{\textbf{\Large I}};
\node at (125.52pt,157.127pt)[below right,color=black]{\textit{Enemy}};
\draw[pstyle2] (97.52pt,182pt) -- (172.0674pt,182pt);
\draw[pstyle2] (97.52pt,190pt) -- (172.0674pt,190pt);
\draw[pstyle0] (455.02pt,280.5pt) arc (180:270:5pt) -- (460.02pt,275.5pt) -- (545.6348pt,275.5pt) arc (270:360:5pt) -- (550.6348pt,280.5pt) -- (550.6348pt,318.5pt) arc (0:90:5pt) -- (545.6348pt,323.5pt) -- (460.02pt,323.5pt) arc (90:180:5pt) -- (455.02pt,318.5pt) -- cycle;
\draw[pstyle1] (470.02pt,291.5pt) ellipse (11pt and 11pt);
\node at (470.02pt,291.5pt)[]{\textbf{\Large I}};
\node at (484.02pt,282.627pt)[below right,color=black]{\textit{Inventory}};
\draw[pstyle2] (456.02pt,307.5pt) -- (549.6348pt,307.5pt);
\draw[pstyle2] (456.02pt,315.5pt) -- (549.6348pt,315.5pt);
\draw[pstyle0] (472.52pt,406pt) arc (180:270:5pt) -- (477.52pt,401pt) -- (528.56pt,401pt) arc (270:360:5pt) -- (533.56pt,406pt) -- (533.56pt,444pt) arc (0:90:5pt) -- (528.56pt,449pt) -- (477.52pt,449pt) arc (90:180:5pt) -- (472.52pt,444pt) -- cycle;
\draw[pstyle1] (487.52pt,417pt) ellipse (11pt and 11pt);
\node at (487.52pt,417pt)[]{\textbf{\Large I}};
\node at (501.52pt,408.127pt)[below right,color=black]{\textit{Item}};
\draw[pstyle2] (473.52pt,433pt) -- (532.56pt,433pt);
\draw[pstyle2] (473.52pt,441pt) -- (532.56pt,441pt);
\draw[pstyle0] (203.02pt,406pt) arc (180:270:5pt) -- (208.02pt,401pt) -- (309.8553pt,401pt) arc (270:360:5pt) -- (314.8553pt,406pt) -- (314.8553pt,444pt) arc (0:90:5pt) -- (309.8553pt,449pt) -- (208.02pt,449pt) arc (90:180:5pt) -- (203.02pt,444pt) -- cycle;
\draw[pstyle1] (218.02pt,417pt) ellipse (11pt and 11pt);
\node at (218.02pt,417pt)[]{\textbf{\Large I}};
\node at (232.02pt,408.127pt)[below right,color=black]{\textit{MagicMove}};
\draw[pstyle2] (204.02pt,433pt) -- (313.8553pt,433pt);
\draw[pstyle2] (204.02pt,441pt) -- (313.8553pt,441pt);
\draw[pstyle0] (155.02pt,12pt) arc (180:270:5pt) -- (160.02pt,7pt) -- (222.4671pt,7pt) arc (270:360:5pt) -- (227.4671pt,12pt) -- (227.4671pt,85.4922pt) arc (0:90:5pt) -- (222.4671pt,90.4922pt) -- (160.02pt,90.4922pt) arc (90:180:5pt) -- (155.02pt,85.4922pt) -- cycle;
\draw[pstyle1] (170.02pt,23pt) ellipse (11pt and 11pt);
\node at (170.02pt,23pt)[]{\textbf{\Large I}};
\node at (184.02pt,14.127pt)[below right,color=black]{\textit{Battle}};
\draw[pstyle2] (156.02pt,39pt) -- (226.4671pt,39pt);
\draw[pstyle2] (156.02pt,47pt) -- (226.4671pt,47pt);
\draw[pstyle3] (166.02pt,61.373pt) ellipse (3pt and 3pt);
\node at (175.02pt,51pt)[below right,color=black]{start()};
\draw[pstyle3] (166.02pt,79.1191pt) ellipse (3pt and 3pt);
\node at (175.02pt,68.7461pt)[below right,color=black]{end()};
\draw[pstyle0] (123.02pt,514pt) arc (180:270:5pt) -- (128.02pt,509pt) -- (247.8851pt,509pt) arc (270:360:5pt) -- (252.8851pt,514pt) -- (252.8851pt,552pt) arc (0:90:5pt) -- (247.8851pt,557pt) -- (128.02pt,557pt) arc (90:180:5pt) -- (123.02pt,552pt) -- cycle;
\draw[color=plantucolor0001,fill=plantucolor0006,line width=1.0pt] (138.02pt,525pt) ellipse (11pt and 11pt);
\node at (138.02pt,525pt)[]{\textbf{\Large E}};
\node at (152.02pt,516.127pt)[below right,color=black]{ElementalType};
\draw[pstyle2] (124.02pt,541pt) -- (251.8851pt,541pt);
\draw[pstyle2] (124.02pt,549pt) -- (251.8851pt,549pt);
\draw[pstyle0] (66.02pt,406pt) arc (180:270:5pt) -- (71.02pt,401pt) -- (162.7353pt,401pt) arc (270:360:5pt) -- (167.7353pt,406pt) -- (167.7353pt,444pt) arc (0:90:5pt) -- (162.7353pt,449pt) -- (71.02pt,449pt) arc (90:180:5pt) -- (66.02pt,444pt) -- cycle;
\draw[pstyle1] (81.02pt,417pt) ellipse (11pt and 11pt);
\node at (81.02pt,417pt)[]{\textbf{\Large I}};
\node at (95.02pt,408.127pt)[below right,color=black]{\textit{TypeTable}};
\draw[pstyle2] (67.02pt,433pt) -- (166.7353pt,433pt);
\draw[pstyle2] (67.02pt,441pt) -- (166.7353pt,441pt);
\draw[pstyle0] (66.02pt,280.5pt) arc (180:270:5pt) -- (71.02pt,275.5pt) -- (150.6771pt,275.5pt) arc (270:360:5pt) -- (155.6771pt,280.5pt) -- (155.6771pt,318.5pt) arc (0:90:5pt) -- (150.6771pt,323.5pt) -- (71.02pt,323.5pt) arc (90:180:5pt) -- (66.02pt,318.5pt) -- cycle;
\draw[pstyle1] (81.02pt,291.5pt) ellipse (11pt and 11pt);
\node at (81.02pt,291.5pt)[]{\textbf{\Large I}};
\node at (95.02pt,282.627pt)[below right,color=black]{\textit{MoveSet}};
\draw[pstyle2] (67.02pt,307.5pt) -- (154.6771pt,307.5pt);
\draw[pstyle2] (67.02pt,315.5pt) -- (154.6771pt,315.5pt);
\draw[pstyle0] (191.02pt,263pt) arc (180:270:5pt) -- (196.02pt,258pt) -- (344.4557pt,258pt) arc (270:360:5pt) -- (349.4557pt,263pt) -- (349.4557pt,336.4922pt) arc (0:90:5pt) -- (344.4557pt,341.4922pt) -- (196.02pt,341.4922pt) arc (90:180:5pt) -- (191.02pt,336.4922pt) -- cycle;
\draw[pstyle1] (233.4124pt,274pt) ellipse (11pt and 11pt);
\node at (233.4124pt,274pt)[]{\textbf{\Large I}};
\node at (253.4996pt,265.127pt)[below right,color=black]{\textit{Attributes}};
\draw[pstyle2] (192.02pt,290pt) -- (348.4557pt,290pt);
\draw[pstyle2] (192.02pt,298pt) -- (348.4557pt,298pt);
\draw[pstyle3] (202.02pt,312.373pt) ellipse (3pt and 3pt);
\node at (211.02pt,302pt)[below right,color=black]{getHealth()};
\draw[pstyle3] (202.02pt,330.1191pt) ellipse (3pt and 3pt);
\node at (211.02pt,319.7461pt)[below right,color=black]{getAttackModifier()};
\draw[pstyle5] (318.1338pt,202.8076pt) ..controls (339.7638pt,217.1376pt) and (354.88pt,231.63pt) .. (367.02pt,258pt) ..controls (382.45pt,291.51pt) and (372.63pt,304.54pt) .. (367.02pt,341pt) ..controls (359.38pt,390.74pt) and (364.59pt,409.81pt) .. (333.02pt,449pt) ..controls (311.56pt,475.66pt) and (279.07pt,495.38pt) .. (250.29pt,508.89pt);
\draw[pstyle6] (308.13pt,196.18pt) -- (310.9227pt,202.8284pt) -- (318.1338pt,202.8076pt) -- (315.3411pt,196.1592pt) -- (308.13pt,196.18pt) -- cycle;
\draw[pstyle5] (89.875pt,205.3659pt) ..controls (70.755pt,219.9759pt) and (58.54pt,233.44pt) .. (48.02pt,258pt) ..controls (14.62pt,336.04pt) and (6pt,375.24pt) .. (48.02pt,449pt) ..controls (64.25pt,477.48pt) and (94.51pt,497.19pt) .. (122.77pt,510.23pt);
\draw[pstyle6] (99.41pt,198.08pt) -- (92.2139pt,198.5446pt) -- (89.875pt,205.3659pt) -- (97.0711pt,204.9013pt) -- (99.41pt,198.08pt) -- cycle;
\draw[pstyle5] (167.5756pt,101.2095pt) ..controls (158.5956pt,121.0195pt) and (153.08pt,133.18pt) .. (145.49pt,149.91pt);
\draw[pstyle6] (172.53pt,90.28pt) -- (166.4096pt,94.0933pt) -- (167.5756pt,101.2095pt) -- (173.696pt,97.3962pt) -- (172.53pt,90.28pt) -- cycle;
\draw[pstyle5] (223.5756pt,100.3892pt) ..controls (236.2456pt,120.1992pt) and (244.55pt,133.18pt) .. (255.26pt,149.91pt);
\draw[pstyle6] (217.11pt,90.28pt) -- (216.9731pt,97.4898pt) -- (223.5756pt,100.3892pt) -- (223.7125pt,93.1794pt) -- (217.11pt,90.28pt) -- cycle;
\draw[pstyle5] (139.1085pt,458.9698pt) ..controls (150.9385pt,476.6298pt) and (160.52pt,490.94pt) .. (172.4pt,508.68pt);
\draw[pstyle5] (132.43pt,449pt) -- (132.446pt,456.2111pt) -- (139.1085pt,458.9698pt) -- (139.0925pt,451.7587pt) -- (132.43pt,449pt) -- cycle;
\draw[pstyle5] (503.02pt,335.73pt) ..controls (503.02pt,357.81pt) and (503.02pt,378.83pt) .. (503.02pt,400.87pt);
\draw[pstyle5] (503.02pt,323.73pt) -- (499.02pt,329.73pt) -- (503.02pt,335.73pt) -- (507.02pt,329.73pt) -- (503.02pt,323.73pt) -- cycle;
\draw[pstyle5] (318.7172pt,200.3943pt) ..controls (350.7572pt,217.1243pt) and (387.14pt,236.22pt) .. (428.02pt,258pt) ..controls (438.6pt,263.63pt) and (450pt,269.76pt) .. (460.63pt,275.5pt);
\draw[pstyle6] (308.08pt,194.84pt) -- (311.5472pt,201.1629pt) -- (318.7172pt,200.3943pt) -- (315.25pt,194.0714pt) -- (308.08pt,194.84pt) -- cycle;
\draw[pstyle5] (230.6426pt,205.5906pt) ..controls (202.2126pt,227.6706pt) and (169.18pt,253.33pt) .. (140.8pt,275.37pt);
\draw[pstyle6] (240.12pt,198.23pt) -- (232.9278pt,198.7512pt) -- (230.6426pt,205.5906pt) -- (237.8348pt,205.0694pt) -- (240.12pt,198.23pt) -- cycle;
\draw[pstyle5] (236.9415pt,458.9698pt) ..controls (225.1115pt,476.6298pt) and (215.53pt,490.94pt) .. (203.64pt,508.68pt);
\draw[pstyle6] (243.62pt,449pt) -- (236.9575pt,451.7587pt) -- (236.9415pt,458.9698pt) -- (243.604pt,456.2111pt) -- (243.62pt,449pt) -- cycle;
\draw[pstyle5] (128.2213pt,210.0097pt) ..controls (123.9313pt,232.0897pt) and (119.8pt,253.33pt) .. (115.52pt,275.37pt);
\draw[pstyle6] (130.51pt,198.23pt) -- (125.4391pt,203.357pt) -- (128.2213pt,210.0097pt) -- (133.2922pt,204.8828pt) -- (130.51pt,198.23pt) -- cycle;
\draw[pstyle5] (148.0735pt,331.4184pt) ..controls (174.5335pt,353.4984pt) and (204.89pt,378.83pt) .. (231.31pt,400.87pt);
\draw[pstyle5] (138.86pt,323.73pt) -- (140.904pt,330.6454pt) -- (148.0735pt,331.4184pt) -- (146.0295pt,324.503pt) -- (138.86pt,323.73pt) -- cycle;
\draw[pstyle5] (270.02pt,210.23pt) ..controls (270.02pt,227pt) and (270.02pt,238.08pt) .. (270.02pt,257.87pt);
\draw[pstyle6] (270.02pt,198.23pt) -- (266.02pt,204.23pt) -- (270.02pt,210.23pt) -- (274.02pt,204.23pt) -- (270.02pt,198.23pt) -- cycle;
\draw[pstyle5] (169.2737pt,206.3302pt) ..controls (187.6037pt,223.1002pt) and (203.98pt,238.08pt) .. (225.61pt,257.87pt);
\draw[pstyle6] (160.42pt,198.23pt) -- (162.1468pt,205.2313pt) -- (169.2737pt,206.3302pt) -- (167.5469pt,199.3289pt) -- (160.42pt,198.23pt) -- cycle;
\end{tikzpicture}
}

	\caption{Schema UML del dominio.} \label{fig:Schema UML del dominio.}
\end{figure}

\chapter{Design}

In questo capitolo si spiegano le strategie messe in campo per soddisfare i requisiti identificati nell'analisi.

Si parte da una visione architetturale, il cui scopo è informare il lettore di quale sia il funzionamento dell'applicativo realizzato ad alto livello.
%
In particolare, è necessario descrivere accuratamente in che modo i componenti principali del sistema si coordinano fra loro.
%
A seguire, si dettagliano alcune parti del design, quelle maggiormente rilevanti al fine di chiarificare la logica con cui sono stati affrontati i principali aspetti dell'applicazione.

\section{Architettura}

Questa sezione spiega come le componenti principali del software interagiscono fra loro.
%
In particolare, qui va spiegato \textbf{se} e \textbf{come} è stato utilizzato il pattern
architetturale model-view-controller (e/o alcune sue declinazioni specifiche, come entity-control-boundary).

Se non è stato utilizzato MVC, va spiegata in maniera molto accurata l'architettura scelta, giustificandola in modo appropriato.

Se è stato scelto MVC, vanno identificate con precisione le interfacce e classi che rappresentano i punti d'ingresso per modello, view, e controller.
Raccomandiamo di sfruttare la definizione del dominio fatta in fase di analisi per capire quale sia l'entry point del model, e di non realizzare un'unica macro-interfaccia che, spesso, finisce con l'essere il prodromo ad una ``God class''.
%
Consigliamo anche di separare bene controller e model, facendo attenzione a non includere nel secondo strategie d'uso che appartengono al primo.

In questa sezione vanno descritte, per ciascun componente architetturale che ruoli ricopre (due o tre ruoli al massimo), ed in che modo interagisce (ossia, scambia informazioni) con gli altri componenti dell'architettura.
%
Raccomandiamo di porre particolare attenzione al design dell'interazione fra view e controller: se ben progettato, sostituire in blocco la view non dovrebbe causare alcuna modifica nel controller (tantomeno nel model).

\subsection*{Elementi positivi}
\begin{itemize}
 \item Si mostrano pochi, mirati schemi UML dai quali si deduce con chiarezza quali sono le parti principali del software e come interagiscono fra loro.
 \item Si mette in evidenza se e come il pattern architetturale model-view-controller è stato applicato, anche con l'uso di un UML che mostri le interfacce principali ed i rapporti fra loro.
 \item Si discute se sia semplice o meno, con l'architettura scelta, sostituire in blocco la view: in un MVC ben fatto, controller e modello non dovrebbero in alcun modo cambiare se si transitasse da una libreria grafica ad un'altra (ad esempio, da Swing a JavaFX, o viceversa).
\end{itemize}

\subsection*{Elementi negativi}
\begin{itemize}
 \item L'architettura è fatta in modo che sia impossibile riusare il modello per un software diverso che affronta lo stesso problema.
 \item L'architettura è tale che l'aggiunta di una funzionalità sul controller impatta pesantemente su view e/o modello.
 \item L'architettura è tale che la sostituzione in blocco della view impatta sul controller o, peggio ancora, sul modello.
 \item Si presentano UML caotici, difficili da leggere.
 \item Si presentano UML in cui sono mostrati elementi di dettaglio non appartenenti all'architettura, ad esempio includenti campi o con metodi che non interessano la parte di interazione fra le componenti principali del software.
 \item Si presentano schemi UML con classi (nel senso UML del termine) che ``galleggiano'' nello schema, non connesse, ossia senza relazioni con il resto degli elementi inseriti.
 \item Si presentano elementi di design di dettaglio, ad esempio tutte le classi e interfacce del modello o della view.
 \item Si discutono aspetti implementativi, ad esempio eventuali librerie usate oppure dettagli di codice.
\end{itemize}

\subsection*{Esempio}

L'architettura di GLaDOS segue il pattern architetturale MVC.
%
Più nello specifico, a livello architetturale, si è scelto di utilizzare MVC in forma ``ECB'', ossia ``entity-control-boundary''\footnote{
Si fa presente che il pattern ECB effettivamente esiste in letteratura come ``istanza'' di MVC, e chi volesse può utilizzarlo come reificazione di MVC.
}.
%
GLaDOS implementa l'interfaccia AI, ed è il controller del sistema.
Essendo una intelligenza artificiale, è una classe attiva.
%
GLaDOS accetta la registrazione di Input ed Output, che fanno parte della ``view'' di MVC, e sono il ``boundary'' di ECB.
Gli Input rappresentano delle nuove informazioni che vengono fornite all'IA, ad esempio delle modifiche nel valore di un sensore, oppure un comando da parte dell'operatore.
Questi input infatti forniscono eventi.
Ottenere un evento è un'operazione bloccante: chi la esegue resta in attesa di un effettivo evento.
Di fatto, quindi, GLaDOS si configura come entità \textit{reattiva}.
Ogni volta che c'è un cambio alla situazione del soggetto, GLaDOS notifica i suoi Output,
informandoli su quale sia la situazione corrente.
%
Conseguentemente, GLaDOS è un ``observable'' per Output.

Con questa architettura, possono essere aggiunti un numero arbitrario di input ed output
all'intelligenza artificiale.
%
Ovviamente, mentre l'aggiunta di output è semplice e non richiede alcuna modifica all'IA, la
presenza di nuovi tipi di evento richiede invece in potenza aggiunte o rifiniture a GLaDOS.
%
Questo è dovuto al fatto che nuovi Input rappresentano di fatto nuovi elementi della business
logic, la cui alterazione od espansione inevitabilmente impatta il controller del progetto.

In \Cref{img:goodarch} è esemplificato il diagramma UML architetturale.


\section{Design dettagliato}

In questa sezione si possono approfondire alcuni elementi del design con maggior dettaglio.
%
Mentre ci attendiamo principalmente (o solo) interfacce negli schemi UML delle sezioni precedenti,
in questa sezione è necessario scendere in maggior dettaglio presentando la struttura di alcune sottoparti rilevanti dell'applicazione.
%
È molto importante che, descrivendo la soluzione ad un problema, quando possibile si mostri che non si è re-inventata la ruota ma si è applicato un design pattern noto.
%
Che si sia utilizzato (o riconosciuto) o meno un pattern noto, è comunque bene definire qual è il problema che si è affrontato, qual è la soluzione messa in campo, e quali motivazioni l'hanno spinta.
%
È assolutamente inutile, ed è anzi controproducente, descrivere classe-per-classe (o peggio ancora metodo-per-metodo) com'è fatto il vostro software: è un livello di dettaglio proprio della documentazione dell'API (deducibile dalla Javadoc).

\textbf{È necessario che ciascun membro del gruppo abbia una propria sezione di design dettagliato,
di cui sarà il solo responsabile}.
%
Ciascun autore dovrà spiegare in modo corretto e giustamente approfondito (non troppo in dettaglio, non superficialmente) il proprio contributo.
%
È importante focalizzarsi sulle scelte che hanno un impatto positivo sul riuso, sull'estensibilità, e sulla chiarezza dell'applicazione.
%
Esattamente come nessun ingegnere meccanico presenta un solo foglio con l'intero progetto di una vettura di Formula 1, ma molteplici fogli di progetto che mostrano a livelli di dettaglio differenti le varie parti della vettura e le modalità di connessione fra le parti, così ci aspettiamo che voi, futuri ingegneri informatici, ci presentiate prima una visione globale del progetto, e via via siate in grado di dettagliare le singole parti, scartando i componenti che non interessano quella in esame.
%
Per continuare il parallelo con la vettura di Formula 1, se nei fogli di progetto che mostrano il
design delle sospensioni anteriori appaiono pezzi che appartengono al volante o al turbo, c'è una
chiara indicazione di qualche problema di design.

Si divida la sezione in sottosezioni, e per ogni aspetto di design che si vuole approfondire, si presenti:
\begin{enumerate}
    \item: una breve descrizione in linguaggio naturale del problema che si vuole risolvere, se necessario ci si può aiutare con schemi o immagini;
    \item: una descrizione della soluzione proposta, analizzando eventuali alternative che sono state prese in considerazione, e che descriva pro e contro della scelta fatta;
    \item: uno schema UML che aiuti a comprendere la soluzione sopra descritta;
    \item: se la soluzione è stata realizzata utilizzando uno o più pattern noti, si spieghi come questi sono reificati nel progetto
    (ad esempio: nel caso di Template Method, qual è il metodo template;
    nel caso di Strategy, quale interfaccia del progetto rappresenta la strategia, e quali sono le sue implementazioni;
    nel caso di Decorator, qual è la classe astratta che fa da Decorator e quali sono le sue implementazioni concrete; eccetera);
\end{enumerate}
%
La presenza di pattern di progettazione \emph{correttamente utilizzati} è valutata molto positivamente.
%
L'uso inappropriato è invece valutato negativamente: a tal proposito, si raccomanda di porre particolare attenzione all'abuso di Singleton, che, se usato in modo inappropriato, è di fatto un anti-pattern.

\subsection*{Elementi positivi}

\begin{itemize}
	\item Ogni membro del gruppo discute le proprie decisioni di progettazione, ed in particolare le azioni volte ad anticipare possibili cambiamenti futuri (ad esempio l'aggiunta di una nuova funzionalità, o il miglioramento di una esistente).
	\item Si mostrano le principali interazioni fra le varie componenti che collaborano alla soluzione di un determinato problema.
	\item Si identificano, utilizzano \textit{appropriatamente}, e descrivono diversi design pattern.
	\item Ogni membro del gruppo identifica i pattern utilizzati nella sua sottoparte.
	\item Si mostrano gli aspetti di design più rilevanti dell'applicazione, mettendo in luce la maniera in cui si è costruita la soluzione ai problemi descritti nell'analisi.
	\item Si tralasciano aspetti strettamente implementativi e quelli non rilevanti, non mostrandoli negli schemi UML (ad esempio, campi privati) e non descrivendoli.
	\item Ciascun elemento di design identificato presenta una piccola descrizione del problema calato
nell'applicazione, uno schema UML che ne mostra la concretizzazione nelle classi del progetto, ed
una breve descrizione della motivazione per cui tale soluzione è stata scelta, specialmente se è stato utilizzato un pattern noto. Ad esempio, se si
dichiara di aver usato Observer, è necessario specificare chi sia l'observable e chi l'observer; se
si usa Template Method, è necessario indicare quale sia il metodo template; se si usa Strategy, è
necessario identificare l'interfaccia che rappresenta la strategia; e via dicendo.
\end{itemize}

\subsection*{Elementi negativi}
\begin{itemize}
	\item Il design del modello risulta scorrelato dal problema descritto in analisi.
	\item Si tratta in modo prolisso, classe per classe, il software realizzato, o comunque si riduce la sezione ad un mero elenco di quanto fatto.
	\item Non si presentano schemi UML esemplificativi.
	\item Non si individuano design pattern, o si individuano in modo errato (si spaccia per design pattern qualcosa che non lo è).
	\item Si utilizzano design pattern in modo inopportuno. Un esempio classico è l'abuso di
Singleton per entità che possono essere univoche ma non devono necessariamente esserlo. Si rammenta
che Singleton ha senso nel secondo caso (ad esempio \texttt{System} e \texttt{Runtime} sono
singleton), mentre rischia di essere un problema nel secondo. Ad esempio, se si rendesse singleton
il motore di un videogioco, sarebbe impossibile riusarlo per costruire un server per partite online
(dove, presumibilmente, si gestiscono parallelamente più partite).
	\item Si producono schemi UML caotici e difficili da leggere, che comprendono inutili elementi di dettaglio.
	\item Si presentano schemi UML con classi (nel senso UML del termine) che ``galleggiano'' nello schema, non connesse, ossia senza relazioni con il resto degli elementi inseriti.
	\item Si tratta in modo inutilmente prolisso la divisione in package, elencando ad esempio le classi una per una.
\end{itemize}

\subsection*{Esempio minimale (e quindi parziale) di sezione di progetto con UML ben realizzati}

\subsubsection{Personalità intercambiabili}

\paragraph{Problema} GLaDOS ha più personalità intercambiabili, la cui presenza deve essere trasparente al client.

\paragraph{Soluzione} Il sistema per la gestione della personalità utilizza il \textit{pattern Strategy}, come da
\Cref{img:strategy}: le implementazioni di \texttt{Personality} possono essere modificate, e la
modifica impatta direttamente sul comportamento di GLaDOS.

\subsubsection{Riuso del codice delle personalità}

\paragraph{Problema} In fase di sviluppo, sono state sviluppate due personalità, una buona ed una cattiva.
Quella buona restituisce sempre una torta vera, mentre quella cattiva restituisce sempre la
promessa di una torta che verrà in realtà disattesa.
Ci si è accorti che diverse personalità condividevano molto del comportamento,
portando a classi molto simili e a duplicazione.

\paragraph{Soluzione} Dato che le due personalità differiscono solo per il comportamento da effettuarsi in caso di percorso completato con successo,
è stato utilizzato il \textit{pattern template method} per massimizzare il riuso, come da \Cref{img:template}.
Il metodo template è \texttt{onSuccess()}, che chiama un metodo astratto e protetto
\texttt{makeCake()}.

\subsubsection{Gestione di output multipli}

\paragraph{Problema} Il sistema deve supportare output multipli. In particolare, si richiede che vi sia un logger che stampa a terminale o su file,
e un'interfaccia grafica che mostri una rappresentazione grafica del sistema.

\paragraph{Soluzione} Dato che i due sistemi di reporting utilizzano le medesime informazioni, si è deciso di raggrupparli dietro l'interfaccia \texttt{Output}.
A questo punto, le due possibilità erano quelle di far sì che \texttt{GLaDOS} potesse pilotarle entrambe.
Invece di fare un sistema in cui questi output sono obbligatori e connessi, si è deciso di usare maggior flessibilità (anche in vista di future estensioni)
e di adottare una comunicazione uno-a-molti fra \texttt{GLaDOS} ed i sistemi di output.
La scelta è quindi ricaduta sul \textit{pattern Observer}: \texttt{GLaDOS} è observable, e le istanze di \texttt{Output} sono observer.
%
Il suo utilizzo è esemplificato in \Cref{img:observer}


\subsection*{Contro-esempio: pessimo diagramma UML}

In \Cref{img:badarch} è mostrato il modo \textbf{sbagliato} di fare le cose.
%
Questo schema è fatto male perché:
\begin{itemize}
	\item È caotico.
	\item È difficile da leggere e capire.
	\item Vi sono troppe classi, e non si capisce bene quali siano i rapporti che intercorrono fra loro.
	\item Si mostrano elementi implementativi irrilevanti, come i campi e i metodi privati nella classe \texttt{AbstractEnvironment}.
	\item Se l'intenzione era quella di costruire un diagramma architetturale, allora lo schema è ancora più sbagliato, perché mostra pezzi di implementazione.
	\item Una delle classi, in alto al centro, galleggia nello schema, non connessa a nessuna altra classe, e di fatto costituisce da sola un secondo schema UML scorrelato al resto
	\item Le interfacce presentano tutti i metodi e non una selezione che aiuti il lettore a capire quale parte del sistema si vuol mostrare.
\end{itemize}


\chapter{Sviluppo}
\section{Testing automatizzato}

Il testing automatizzato è un requisito di qualunque progetto software che si rispetti, e consente di verificare che non vi siano regressioni nelle funzionalità a fronte di aggiornamenti.
%
Per quanto riguarda questo progetto è considerato sufficiente un test minimale, a patto che sia completamente automatico.
%
Test che richiedono l'intervento da parte dell'utente sono considerati \textit{negativamente} nel computo del punteggio finale.

\subsection*{Elementi positivi}

\begin{itemize}
 \item Si descrivono molto brevemente i componenti che si è deciso di sottoporre a test automatizzato.
 \item Si utilizzano suite specifiche (e.g. JUnit) per il testing automatico.
\end{itemize}

\subsection*{Elementi negativi}
\begin{itemize}
 \item Non si realizza alcun test automatico.
 \item La non presenza di testing viene aggravata dall'adduzione di motivazioni non valide. Ad esempio, si scrive che l'interfaccia grafica non è testata automaticamente perché è \emph{impossibile} farlo\footnote{Testare in modo automatico le interfacce grafiche è possibile (si veda, come esempio, \url{https://github.com/TestFX/TestFX}), semplicemente nel corso non c'è modo e tempo di introdurvi questo livello di complessità. Il fatto che non vi sia stato insegnato come farlo non implica che sia impossibile!}.
 \item Si descrive un testing di tipo manuale in maniera prolissa.
 \item Si descrivono test effettuati manualmente che sarebbero potuti essere automatizzati, ad esempio scrivendo che si è usata l'applicazione manualmente.
 \item Si descrivono test non presenti nei sorgenti del progetto.
 \item I test, quando eseguiti, falliscono.
\end{itemize}

\section{Note di sviluppo}

Questa sezione, come quella riguardante il design dettagliato va svolta \textbf{singolarmente da ogni membro del gruppo}.
%
Nella prima parte, ciascuno dovrà mostrare degli esempi di codice particolarmente ben realizzati,
che dimostrino proefficienza con funzionalità avanzate del linguaggio e capacità di spingersi oltre le librerie mostrate a lezione.

\begin{itemize}
	\item \textbf{Elencare} (fare un semplice elenco per punti, non un testo!) le feature \textit{avanzate} del linguaggio e dell'ecosistema Java che sono state
utilizzate. Le feature di interesse sono:
	\begin{itemize}
		\item Progettazione con generici, ad esempio costruzione di nuovi tipi generici, e uso di generici bounded.
		L'uso di classi generiche di libreria non è considerato avanzato.
		\item Uso di lambda expressions
		\item Uso di \texttt{Stream}, di \texttt{Optional} o di altri costrutti funzionali
		\item Uso di reflection
		\item Definizione ed uso di nuove annotazioni
		\item Uso del Java Platform Module System
		\item Uso di parti della libreria JDK non spiegate a lezione (networking, compressione, parsing XML, eccetera...)
		\item Uso di librerie di terze parti (incluso JavaFX): Google Guava, Apache Commons...
	\end{itemize}
	\item Si faccia molta attenzione a non scrivere banalità, elencando qui features di tipo ``core'', come le eccezioni, le enumerazioni, o le inner class: nessuna di queste è considerata avanzata.
	\item Per ogni feature avanzata, mostrata, includere:
	\begin{itemize}
		\item Nome della feature
		\item Permalink GitHub al punto nel codice in cui è stata utilizzata
	\end{itemize}
\end{itemize}

In questa sezione, \textit{dopo l'elenco},
vanno menzionati ed attributi con precisione eventuali pezzi di codice ``riadattati'' (o scopiazzati...) da Internet o da altri progetti,
pratica che tolleriamo ma che non raccomandiamo.
%
Si rammenta agli studenti che non è consentito partire da progetti esistenti e procedere per modifiche successive.
%
Si ricorda anche che i docenti hanno in mano strumenti antiplagio piuttosto raffinati e che ``capiscono'' il codice e la storia delle modifiche del progetto,
per cui tecniche banali come cambiare nomi (di classi, metodi, campi, parametri, o variabili locali),
aggiungere o togliere commenti,
oppure riordinare i membri di una classe vengono individuate senza problemi.
%
Le regole del progetto spiegano in dettaglio l'approccio dei docenti verso atti gravi come il plagiarismo.

I pattern di design \textbf{non} vanno messi qui.
%
L'uso di pattern di design (come suggerisce il nome) è un aspetto avanzato di design, non di implementazione,
e non va in questa sezione.

\subsection*{Elementi positivi}

\begin{itemize}
	\item Si elencano gli aspetti avanzati di linguaggio che sono stati impiegati
	\item Si elencano le librerie che sono state utilizzate
	\item Per ciascun elemento, si fornisce un permalink
	\item Ogni permalink fa riferimento ad uno snippet di codice scritto dall'autore della sezione (i docenti verificheranno usando \texttt{git blame})
	\item Se si è utilizzato un particolare algoritmo, se ne cita la fonte originale.
	Ad esempio, se si è usato Mersenne Twister per la generazione di numeri pseudo-random, si cita \cite{mersenne}.
	\item Si identificano parti di codice prese da altri progetti, dal web, o comunque scritte in forma originale da altre persone.
	In tal senso, si ricorda che agli ingegneri non è richiesto di re-inventare la ruota continuamente:
	se si cita debitamente la sorgente è tollerato fare uso di di snippet di codice open source per risolvere velocemente problemi non banali.
	Nel caso in cui si usino snippet di codice di qualità discutibile,
	oltre a menzionarne l'autore originale si invitano gli studenti ad adeguare tali parti di codice agli standard e allo stile del progetto.
	Contestualmente, si fa presente che è largamente meglio fare uso di una libreria che copiarsi pezzi di codice:
	qualora vi sia scelta (e tipicamente c'è), si preferisca la prima via.
\end{itemize}

\subsection*{Elementi negativi}
\begin{itemize}
	\item Si elencano feature core del linguaggio invece di quelle segnalate. Esempi di feature core da non menzionare sono:
    \begin{itemize}
        \item eccezioni;
        \item classi innestate;
        \item enumerazioni;
        \item interfacce.
    \end{itemize}
	\item Si elencano applicazioni di terze parti (peggio se per usarle occorre licenza, e lo studente ne è sprovvisto) che non c'entrano nulla con lo sviluppo, ad esempio:
    \begin{itemize}
        \item Editor di grafica vettoriale come Inkscape o Adobe Illustrator;
        \item Editor di grafica scalare come GIMP o Adobe Photoshop;
        \item Editor di audio come Audacity;
        \item Strumenti di design dell'interfaccia grafica come SceneBuilder: il codice è in ogni caso inteso come sviluppato da voi.
    \end{itemize}
	\item Si descrivono aspetti di scarsa rilevanza, o si scende in dettagli inutili.
	\item Sono presenti parti di codice sviluppate originalmente da altri che non vengono debitamente segnalate.
	In tal senso, si ricorda agli studenti che i docenti hanno accesso a tutti i progetti degli anni passati,
	a Stack Overflow,
	ai principali blog di sviluppatori ed esperti Java,
	ai blog dedicati allo sviluppo di soluzioni e applicazioni
	(inclusi blog dedicati ad Android e allo sviluppo di videogame),
	nonché ai vari GitHub, GitLab, e Bitbucket.
	Conseguentemente, è \emph{molto} conveniente \emph{citare} una fonte ed usarla invece di tentare di spacciare per proprio il lavoro di altri.
	\item Si elencano design pattern
\end{itemize}

\subsection{Esempio}

\subsubsection{Utilizzo della libreria SLF4J}

Utilizzata in vari punti.
Un esempio è \url{https://github.com/AlchemistSimulator/Alchemist/blob/5c17f8b76920c78d955d478864ac1f11508ed9ad/alchemist-swingui/src/main/java/it/unibo/alchemist/boundary/swingui/effect/impl/EffectBuilder.java#L49}

\subsubsection{Utilizzo di \texttt{LoadingCache} dalla libreria Google Guava}

Permalink: \url{https://github.com/AlchemistSimulator/Alchemist/blob/d8a1799027d7d685569e15316a32e6394632ce71/alchemist-incarnation-protelis/src/main/java/it/unibo/alchemist/protelis/AlchemistExecutionContext.java#L141-L143}

\subsubsection{Utilizzo di \texttt{Stream} e lambda expressions}

Usate pervasivamente. Il seguente è un singolo esempio.
Permalink: \url{https://github.com/AlchemistSimulator/Alchemist/blob/d8a1799027d7d685569e15316a32e6394632ce71/alchemist-incarnation-protelis/src/main/java/it/unibo/alchemist/model/ProtelisIncarnation.java#L98-L120}

\subsubsection{Scrittura di metodo generico con parametri contravarianti}

Permalink: \url{https://github.com/AlchemistSimulator/Alchemist/blob/d8a1799027d7d685569e15316a32e6394632ce71/alchemist-incarnation-protelis/src/main/java/it/unibo/alchemist/protelis/AlchemistExecutionContext.java#L141-L143}

\subsubsection{Protezione da corse critiche usando \texttt{Semaphore}}

Permalink: \url{https://github.com/AlchemistSimulator/Alchemist/blob/d8a1799027d7d685569e15316a32e6394632ce71/alchemist-incarnation-protelis/src/main/java/it/unibo/alchemist/model/ProtelisIncarnation.java#L388-L440}


\chapter{Commenti finali}

In quest'ultimo capitolo si tirano le somme del lavoro svolto e si delineano eventuali sviluppi
futuri.

\textit{Nessuna delle informazioni incluse in questo capitolo verrà utilizzata per formulare la valutazione finale}, a meno che non sia assente o manchino delle sezioni obbligatorie.
%
Al fine di evitare pregiudizi involontari, l'intero capitolo verrà letto dai docenti solo dopo aver formulato la valutazione.

\section{Autovalutazione e lavori futuri}

\textbf{È richiesta una sezione per ciascun membro del gruppo, obbligatoriamente}.
%
Ciascuno dovrà autovalutare il proprio lavoro, elencando i punti di forza e di debolezza in quanto prodotto.
Si dovrà anche cercare di descrivere \emph{in modo quanto più obiettivo possibile} il proprio ruolo all'interno del gruppo.
Si ricorda, a tal proposito, che ciascuno studente è responsabile solo della propria sezione: non è un problema se ci sono opinioni contrastanti, a patto che rispecchino effettivamente l'opinione di chi le scrive.
Nel caso in cui si pensasse di portare avanti il progetto, ad esempio perché effettivamente impiegato, o perché sufficientemente ben riuscito da poter esser usato come dimostrazione di esser capaci progettisti, si descriva brevemente verso che direzione portarlo.

\section{Difficoltà incontrate e commenti per i docenti}

Questa sezione, \textbf{opzionale}, può essere utilizzata per segnalare ai docenti eventuali problemi o difficoltà incontrate nel corso o nello svolgimento del progetto, può essere vista come una seconda possibilità di valutare il corso (dopo quella offerta dalle rilevazioni della didattica) avendo anche conoscenza delle modalità e delle difficoltà collegate all'esame, cosa impossibile da fare usando le valutazioni in aula per ovvie ragioni.
%
È possibile che alcuni dei commenti forniti vengano utilizzati per migliorare il corso in futuro: sebbene non andrà a vostro beneficio, potreste fare un favore ai vostri futuri colleghi.
%
Ovviamente \textit{il contenuto della sezione non impatterà il voto finale}.

\appendix
\chapter{Guida utente}

Capitolo in cui si spiega come utilizzare il software. Nel caso in cui il suo uso sia del tutto
banale, tale capitolo può essere omesso.
%
A tal riguardo, si fa presente agli studenti che i docenti non hanno mai utilizzato il software
prima, per cui aspetti che sembrano del tutto banali a chi ha sviluppato l'applicazione possono non
esserlo per chi la usa per la prima volta.
%
Se, ad esempio, per cominciare una partita con un videogioco è necessario premere la barra
spaziatrice, o il tasto ``P'', è necessario che gli studenti lo segnalino.

\subsection*{Elementi positivi}

\begin{itemize}
 \item Si istruisce in modo semplice l'utente sull'uso dell'applicazione, eventualmente facendo uso di schermate e descrizioni.
\end{itemize}

\subsection*{Elementi negativi}
\begin{itemize}
 \item Si descrivono in modo eccessivamente minuzioso tutte le caratteristiche, anche minori, del software in oggetto.
 \item Manca una descrizione che consenta ad un utente qualunque di utilizzare almeno le funzionalità primarie dell'applicativo.
\end{itemize}

\chapter{Esercitazioni di laboratorio}

In questo capitolo ciascuno studente elenca gli esercizi di laboratorio che ha svolto
(se ne ha svolti),
elencando i permalink dei post sul forum dove è avvenuta la consegna.
%
Questa sezione potrebbe essere processata da strumenti automatici,
per cui link a oggetti diversi dal permalink della consegna,
errori nell'email o nel nome del laboratorio possono portare ad ignorare alcune consegne,
si raccomanda la massima precisione.

\section*{Esempio}

\subsection{paolino.paperino@studio.unibo.it}

\begin{itemize}
 \item Laboratorio 04: \url{https://virtuale.unibo.it/mod/forum/discuss.php?d=12345#p123456}
 \item Laboratorio 06: \url{https://virtuale.unibo.it/mod/forum/discuss.php?d=22222#p222222}
 \item Laboratorio 09: \url{https://virtuale.unibo.it/mod/forum/discuss.php?d=99999#p999999}
\end{itemize}

\subsection{paperon.depaperoni@studio.unibo.it}

\begin{itemize}
 \item Laboratorio 04: \url{https://virtuale.unibo.it/mod/forum/discuss.php?d=12345#p123456}
 \item Laboratorio 05: \url{https://virtuale.unibo.it/mod/forum/discuss.php?d=22222#p222222}
 \item Laboratorio 06: \url{https://virtuale.unibo.it/mod/forum/discuss.php?d=99999#p999999}
 \item Laboratorio 07: \url{https://virtuale.unibo.it/mod/forum/discuss.php?d=22222#p222222}
 \item Laboratorio 08: \url{https://virtuale.unibo.it/mod/forum/discuss.php?d=99999#p999999}
 \item Laboratorio 09: \url{https://virtuale.unibo.it/mod/forum/discuss.php?d=22222#p222222}
 \item Laboratorio 10: \url{https://virtuale.unibo.it/mod/forum/discuss.php?d=99999#p999999}
 \item Laboratorio 11: \url{https://virtuale.unibo.it/mod/forum/discuss.php?d=22222#p222222}
\end{itemize}


\bibliographystyle{alpha}
\bibliography{13-template}

\end{document}